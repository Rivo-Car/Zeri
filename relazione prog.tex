\documentclass[a4paper,11pt]{article}
\usepackage[T1]{fontenc}
\usepackage[italian]{babel}

 

\begin{document}
\title{Relazione progetto di programmazione: calcolo degli zeri di una funzione }
\author{Giulia Altobel, Nicole Corso, Alessio Gianello, Rivo  }


\maketitle
 \section {Introduzione}




\section{Metodo di Bisezione}
\subsection{Ipotesi di applicazione:}

Per  poter applicare il metodo di Bisezione è necessario che le ipotesi del teorema seguente siano soddisfatte: 
\\
\\
 \textit{Teorma di Bolzano:}
Siano a e b due numeri reali tali che:

\begin{enumerate}
\item  f(a)f(b)<0;
\item  f sia continua su [a,b]
\end{enumerate}	

allora esiste almeno un $\bar{x}$ $\in$  (a, b) tale che f($\bar{x}$)=0.
\\
\subsection{Procedimento:}
Si procede dividendo l'intervallo in due parti uguali e calcolando il valore della funzione nel punto medio di ascissa m = (a+b)/2.
Se risulta f(m) = f((a+b)/2) = 0 allora m è la radice cercata; altrimenti tra i due intervalli [a, m) e (m, b] si sceglie quello ai cui estremi la funzione  assume 	valori di segno opposto. Si ripete per questo  nuovo intervallo la procedura di dimezzamento.
Così continuando si ottiene una successione di intervalli [a1, b1], [a2, b2], ..., [an, bn] incapsulati, cioè ognuno incluso nel precedente. Questi intervalli hanno come ampiezze
\begin{equation} bn - an = (b-a) / 2^n  \end {equation}per n =1,2

I valori $a_i$ sono valori approssimati per difetto della radice, i valori di $b_i$ sono invece i valori della radice approssimati per eccesso. 
Gli $a_n$ formano una successione crescente limitata ed i $b_n$ formano una successione decrescente limitata.
Le due successioni ammettono lo stesso limite che è la radice dell'equazione esaminata.

\subsection{Criterio di arresto ed errore:}

L'algoritmo viene arrestato quando f(m) è abbastanza vicino a 0 e/o si raggiunge il numero massimo di iterazioni consentite .
Dunque come stima della soluzione (alphasimbolo)  alla fine avremo un certo m. 

\subsection {Analisi:}
\subsubsection{Vantaggi:}
\begin{enumerate}
\item Vale la seuente relazione:
%\begin{equation}	lim (n->(freccietta per dire che il limite tende a + infinito)+inf) |e_n| = 0 \end{equation}
quindi la convergenza del metodo è globale (converge sempre); è quindi un metodo molto robusto.
\item  Ha il notevole pregio di essere stabile in ogni occasione e quindi di garantire sempre la buona riuscita dell'operazione.
\item Dal p.d.v. del costo computazionale, poichè ad ogni iterazione si esegue una sola valutazione di funzione, il costo computazionale risulta essere basso. 

\end{enumerate}

\subsubsection{Svantaggi:}
\begin {enumerate}
\item L'efficienza del metodo di bisezione è scarsa e sono richieste ipotesi particolarmente restrittive;
\item Se richiediamo \begin {equation}|e_n| <= tolleranza  \end {equation}otteniamo la seguente condizione per n:
\begin{equation}	n >= log_2 ((b-a) / tollerr) - 1. \end{equation}
Essendo  %log_2 10 =3.32
 (uguale con le tilde) servono in media più di tre bisezioni per migliorare di una cifra significativa l'accuratezza della radice, quindi la convergenza è  lenta;
\item Inoltre la riduzione dell'errore a ogni passaggio non è monotona, cioè non è detto che:  \begin {equation}|e_n+1| < |e_n | \end{equation}per ogni n = 1,2\dots
\item Il metodo funziona solo se la funzione ammette un solo zero e non può calcolare zeri complessi.
\end{enumerate}
\section {Metodo di Newton o delle tangenti}
\subsection{Ipotesi:}
Il metodo delle tangenti, chiamato anche metodo di Newton o metodo di Newton-Raphson, è uno dei metodi per il calcolo approssimato di una soluzione di 		un'equazione della forma f(x)=0.
\\
\\
\textit{Teorema degli zeri:}
La condizione necessaria affinché il metodo sia applicabile è che esista un intervallo [a,b] in cui x0 nell'intervallo è tale che f'(x0) (diverso) 0 e f''(x0) (diverso) 0 		
\subsection{Procedimento:}
Il metodo consiste nel sostituire alla curva y=f(x) la tangente alla curva stessa, partendo da un qualsiasi punto: considerando l’equazione del fascio di rette:
\begin{equation} y-f(x0)=m(x-x0)\end{equation} posto y=0 e m=f'(x0), x1 sarà:\begin{equation} x1=x0-(f(x0)/f'(x0)) \end{equation}.
Generalizzando all’iterazione k: $x_k-1$ =$x_k$ - (f($x_k$))/f'($x_k$))), che è la relazione di ricorrenza del metodo (che permette di determinare successive 	approssimazioni della radice dell'equazione y=f(x)=0).
\subsection{Criterio di arresto ed errore:}
L'algoritmo viene arrestato quando viene raggiunto il massimo numero di iterazioni e/o quando l'errore risulta minore della tolleranza prestabilita.
L'errore è dato dalla seguente relazione:
\begin{equation} e_n=|x_1-x| \end{equation}
Dove con x si intende l'itererazione precedente e con $x_1$ si intende l'iterazione corrente.
\subsection{Analisi:}
\subsubsection{Vantaggi:}
Con le ipotesi poste, si dimostra che la successione delle $x_n$ converge alla radice piuttosto rapidamente.
La convergenza è quadratica, cioè il numero di cifre significative approssimativamente raddoppia ad ogni iterazione.
\subsubsection{Svantaggi:}
\begin{enumerate}
\item Il metodo è applicabile solo a funzioni derivabili con derivata prima diversa da zero nei punti della successione;
\\
\item La convergenza non è garantita, in particolare quando f'(x) varia notevolmente in prossimità dello zero. 
Inoltre, il metodo assume che f'(x) sia disponibile direttamente per un dato x. Nei casi in cui questo non si verifichi e risultasse necessario calcolare la derivata attraverso una differenza finita, è consigliabile usare il metodo delle secanti;
\\
\item Il numero di iterazioni dipende principalmente dalla posizione del punto iniziale rispetto allo zero. 
In base alla funzione analizzata e al valore di x0 assunto come prima iterazione il metodo può convergere o meno. Il metodo risulta meno solido rispetto al metodo di bisezione.
\end{enumerate}
\subsubsection{Confronto con Bisezione:}
\begin{enumerate}
\item Si può subito notare la velocità di convergenza del metodo rispetto al metodo precedente: col metodo di bisezione cresce linearmente, anche se solo localmente (cioè non vale per ogni intervallo).
\item Il metodo esegue un numero di iterazioni significativamente inferiore. 
\item  Come costo computazionale, ad ogni iterazione esegue due valutazioni di funzione. È quindi più pesante rispetto al precedente metodo. 

\end{enumerate}

\end {document}